\documentclass[a4paper,11pt]{article}
\usepackage{amsmath}
\usepackage{amsfonts}
\usepackage{amssymb}
\usepackage{graphicx}
\usepackage[left=2cm,right=2cm,top=2cm,bottom=2cm]{geometry}
\usepackage[utf8]{inputenc}
\usepackage[spanish,es-sloppy]{babel}
\author{HeNeos}
\title{Bits Mask \& Next permutation}
\usepackage{verbments}
\definecolor{fondo1}{rgb}{0.9764, 0.9764, 0.9762}
\definecolor{fondo2}{rgb}{0.1647, 0.4980, 0.7}
\begin{document}
%bgcolor=fondo del codigo
%captionbgcolor=fondo del nombre del codigo
%numbersep= separacion de los numeros al codigo
%texcl= activa los comentarios
%styles, introducir el codigo en la terminal: pygmentize -L styles
\fvset{frame=bottomline, framerule=0.02cm,numbers=left, numbersep=8pt}
\plset{language=c++,texcl=true, listingnamefont=\sffamily\bfseries\color{white},captionbgcolor=fondo2, bgcolor=fondo1,listingname=\textbf{Programa}, captionfont=\sffamily\color{white}}
\maketitle
\section{PC-UNI}
\subsection*{Objetivos}
\begin{itemize}
\item Analizar problemas que requieran computar todas las permutaciones de un arreglo
\item Analizar problemas que requieran computar todos los subconjuntos de un arreglo
\item Mostrar como hacer sobrecarga de operadores
\end{itemize}
\subsection*{Problema motivacinal 1}
Dado un entero $n$. Imprimir todas las permutaciones de \{1,2,3,$\ldots${,}$n$\}. Límites: 0 $\leq n \leq$ 9.\\[20pt]
Una solución a este problema es simplemente usar la STL. Pues, hay funciones que nos pueden ayudar a hacer esta tarea:
\begin{itemize}
\item \texttt{next\_permutation}
\item \texttt{iota}
\end{itemize}
Obteniendo una solución $O(n\cdot n!)$
\begin{pyglist}[language=c++,caption={Problema motivacional 1},style=manni]
#include <bits/stdc++.h>

#define all(X) begin(X), end(X)

using namespace std;

int main () {
  int n;
  cin >> n;
  vector <int> p(n);
  iota(all(p), 1);
  do {
    for (int p_i: p) cout << p_i << ' ';
    cout << endl;
  } while(next_permutation(all(p)));
  return (0);
}
//Codigo hecho con style=manni
\end{pyglist}
\subsection*{Problema motivacional 2}
Dado un entero $n$ seguido de $n$ enteros: $a_{1}, a_{2}, a_{3}, \ldots, a_{n}$.$\!$ Imprimir la suma de elementos de cada subconjunto de los $n$ dados. Límites: 1 $\leq n \leq$ 20.\\[20pt]
Una solución a este problema es usando máscara de bits. Para entender ello, primero veamos algunos datos útiles:
\begin{itemize}
\item Un \texttt{int} usa 32-bit
\item Un \texttt{long long} usa 64-bit
\item Un entero que usa $k$ bits puede almacenar números en $[-2^{k-1},2^{k-1})$
\item Un entero sin singo (\texttt{unsigned}) que usa $k$ bits puede almacenar números en $[0,2^{k})$
\end{itemize}
En \texttt{C++} podemos comprobar lo anterior con el siguiente código:
\begin{pyglist}[language=c++,caption={Comprobación},listingname={\textbf{Ejemplo}},style=igor]
cout << INT_MIN << ' ' << INT_MAX << endl;
cout << UINT_MAX << endl;
cout << LLONG_MIN << ' ' << LLONG_MAX << endl;
cout << ULLONG_MAX << endl;
//Codigo hecho con style=igor
\end{pyglist}
Los enteros se guardan en su representación binaria usando $k$ bits (completando con 0's si es necesario).\\
Por ejemplo, para un entero de 8 bits, los siguientes números se guardarían así:
\begin{itemize}
\item 5=00000101
\item 7=00000111
\item 8=00001000
\item 12=00001100
\item 19=00010011
\end{itemize}
En \texttt{C++} podemos comprobar lo anterior con el siguiente código:
\begin{pyglist}[language=c++,caption={Comprobación},listingname={\textbf{Ejemplo}},style=lovelace]
for (int num: {5, 7, 8, 12, 19}) {
    cout << setw(2) << num << ' ' << bitset <8>(num) << endl;
  }
//Codigo hecho con style=lovelace
\end{pyglist}
En \texttt{C++} podemos trabajar con los números a nivel de bits usando estas operaciones:
\begin{itemize}
\item $\&$
\item $\vert$
\item $\wedge$
\item $\sim$
\end{itemize}
Además en \texttt{C++} podemos mover todos los bits de un número hacia la izquierda o la derecha con los operadores \texttt{<<} y \texttt{>>} respectivamente.\\
Con esto, podemos:
\begin{description}
\item[Obtener $2^{k}$:] \texttt{1 << k}
\item[Alternar el $k$-esimo bit:] \texttt{x $\wedge$ (1 << k)}
\item[Apagar el $k$-esimo bit:] \texttt{x \& ($\sim$(1 << k))}
\item[Prender el $k$-esimo bit:] \texttt{x $\vert$ (1 << k)}
\item[Obtener el $k$-esimo bit:] \texttt{(x >> k) \& 1}
\end{description}
Así, podemos resolver nuestro problema motivacional en $O(n\cdot 2^{n})$ con el siguiente código:
\begin{pyglist}[language=c++,caption={Problema motivacional 2},style=xcode]
int n;
  cin >> n;
  vector <int> arr(n);
  for (int i = 0; i < n; i++) cin >> arr[i];
  for (int mask = 0; mask < (1 << n); mask++) {
    int sum = 0;
    for (int bit = 0; bit < n; bit++) {
      if ((mask >> bit) & 1) sum += arr[bit];
    }
    cout << bitset <20> (mask) << " suma = " << sum << endl;
  }
//Codigo hecho con style=xcode
\end{pyglist}
Lo anterior también te permitiría resolver (con un poco de ingenio) algunas problemas que requieran manipular bits.\\
También te podría interesar:
\begin{itemize}
\item \texttt{\_builtin\_popcount}
\item \texttt{\_builtin\_clz}
\item \texttt{\_builtin\_ctz}
\item ¿Cómo se guardan los números negativos?
\item La clase de manejo de bits del año pasado
\end{itemize}
\newpage
\subsection*{Problema motivacional 3}
Dado un entero $n$ seguido de $n$ pares de elementos $a_{i}, b_{i}$ (que representan a la fracción $a_{i}/b_{i}$). Imprimir las $n$ fracciones en forma creciente. Límites: $1 \leq n \leq 10^{5}$.\\[20pt]
Iremos explicando en clase como llegar a codear esto:
\begin{pyglist}[language=c++,caption={Prueba},style=vs]
/**
 * Ejemplo de como crear un 'struct' y como usarlo
 * Problema
 * - Queremos tener un tipo de dato de represente una fraccion
 * - Queremos poder ordenar fracciones siguiendo esta relacion de orden
 *      a1   a2
 *      -- < --      <->     a1 * b2 < a2 * b1
 *      b1   b2
 * - Queremos poder usar la operacion '+' y '*' entre fracciones
 * Recibire un numero 'n' seguido de 'n' fracciones
 * Debo imprimir
 * Las fracciones ordenadas
 * La suma de las fracciones
 * El producto de las fracciones
 **/
// Para simplificar la solucion
// aceptare que las fracciones no tienen 0 en el denominador
#include <bits/stdc++.h>
using namespace std;
struct Fraccion {
    int num, den;
    // Los constructores tienen el mismo nombre que mi estructura
    // Puedo tener mas de un constructor
    // Estos se diferenciaran por los parametros que reciban
    Fraccion() {}
    Fraccion(int x, int y) {
        num = x;
        den = y;
    }
    // Puedo sobreescribir el operador '<' para poder usar la funcion 'sort'
    // - (Fraccion\& otra)
    //   Indica que esta funcion recibira una referencia de la variable que invoque
    //   este metodo. Asi, no se creara una copia de esa variable para esta funcion
    //   Lo que se haga con esta variable dentro de la funcion se vera reflejado en
    //   la variable que la invoco
    // - (const Fraccion otra)
    //   Indica que el valor de la variable 'otra' sera constante en ese metodo
    // - () const \{
    //   Indica que este metodo no cambiara el estado de ningun atributo de
    //   la instancia de esta 'struct' que invoque el metodo
    bool operator < (const Fraccion& otra) const {
        return num * otra.den < den * otra.num;
    }
    // Si deseo, puedo no usar 'const' y '\&'
    // Pero esto sera un poco mas lento ya que crea copias innecesarias
    // Y al no definir que es un 'const', no da libertad al compilador de
    // hacer optimizaciones
    Fraccion operator + (Fraccion otra) {
        return Fraccion(num * otra.den + den * otra.num, den * otra.den);
    }
    // Este operador cambiara el estado de la instancia que la invoque
    // por ello no podemos usar () const \{
    void operator *= (const Fraccion& otra){
        num = num * otra.num;
        den = den * otra.den;
    }
    // Tambien puedo escribir funciones ('metodos') para este 'struct'
    void imprimir(string sep) {
        cout << num << '/' << den << sep;
    }
}; // NO OLVIDAR PONER ';' al final de un 'struct'
int n;
vector <Fraccion> arr;
// Si no deseo sobreescribir el operador ('<') en mi 'struct', puedo definirlo
// como una funcion, asi:
bool cmp(const Fraccion& X, const Fraccion& Y) {
    return X.num * Y.den < X.den * Y.num;
}
int main() {
    cin >> n;
    for (int i = 0; i < n; i++) {
        int num, den;
        cin >> num >> den;
        arr.push_back(Fraccion(num, den));
    }
    sort(arr.begin(), arr.end());
    //sort(arr.begin(), arr.end(), cmp);
    cout << "Las fracciones ordenadas" << endl;
    for (auto fraccion : arr) fraccion.imprimir("  ");
    cout << endl;
    Fraccion suma = Fraccion(0, 1);
    for (auto fraccion : arr) {
        suma = suma + fraccion;
        // Si quisiera usar suma += fraccion
        // Tendria que definir el operador '+='
    }
    cout << "La suma de las fracciones es" << endl;
    suma.imprimir("\n");
    Fraccion producto = Fraccion(1, 1);
    for (auto fraccion : arr) producto *= fraccion;
    cout << "El producto de las fracciones es" << endl;
    producto.imprimir("\n");

    return (0);
}
//Codigo hecho con style=vs
\end{pyglist}
Te podría interesar investigar sobre: \texttt{stable\_sort}
\section{Investigación}
\subsection*{\texttt{next\_permutation}}
Se utiliza para reorganizar los elementos en el rango \texttt{[first, last)} en la siguiente permutación lexicográficamente mayor. La función es del tipo \texttt{bool}, por lo que solo devuelve \texttt{true} o \texttt{false}.
\begin{pyglist}[language=c++,caption={next\_permutation},listingname={\textbf{Ejemplo}},style=autumn]
#include <algorithm>  

#include <iostream> 

using namespace std; 

int main() { 
    int arr[] = { 1, 2, 3 }; 
    sort(arr, arr + 3); 
  
    cout << "The 3! possible permutations with 3 elements:\n"; 
    do { 
        cout << arr[0] << " " << arr[1] << " " << arr[2] << "\n"; 
    } while (next_permutation(arr, arr + 3)); 

    cout << "After loop: " << arr[0] << ' ' 
         << arr[1] << ' ' << arr[2] << '\n'; 

    return 0; 
} 
//Salida:
//The 3! possible permutations with 3 elements:
//1 2 3
//1 3 2
//2 1 3
//2 3 1
//3 1 2
//3 2 1
//After loop: 1 2 3

//Codigo hecho con style=autumn
\end{pyglist}
\newpage
\subsection*{\texttt{prev\_permutation}}
De la misma forma que \texttt{next\_permutation} solo que devuelve una permutación anterior.
\begin{pyglist}[language=c++,caption={prev\_permutation},listingname={\textbf{Ejemplo}},style=abap]
#include <algorithm>  

#include <iostream> 

using namespace std; 

int main() { 
    int arr[] = { 1, 2, 3 }; 
    sort(arr, arr + 3); 
    reverse(arr, arr + 3); 
 
    cout << "The 3! possible permutations with 3 elements:\n"; 
    do { 
        cout << arr[0] << " " << arr[1] << " " << arr[2] << "\n"; 
    } while (prev_permutation(arr, arr + 3)); 

    cout << "After loop: " << arr[0] << ' ' << arr[1]  
         << ' ' << arr[2] << '\n'; 

    return 0; 
} 
//Salida:
//The 3! possible permutations with 3 elements:
//3 2 1
//3 1 2
//2 3 1
//2 1 3
//1 3 2
//1 2 3
//After loop: 3 2 1

//Codigo hecho con style=abap
\end{pyglist}
\newpage
\subsection*{\texttt{iota}}
Asigna a cada elemento en el rango \texttt{[first,last)} valores sucesivos de \texttt{val}, como si se tratará de \texttt{val++}.
\begin{pyglist}[language=c++,caption={iota},listingname={\textbf{Ejemplo}},style=rrt]
#include <iostream> 
#include <numeric>
using namespace std;  
  
int main() { 
    int numbers[10]; 
    // Initailising starting value as 100 
    int st = 100; 
  
    iota(numbers, numbers + 10, st); 
  
    cout << "Elements are :"; 
    for (auto i : numbers) 
        cout << ' ' << i; 
    cout << '\n'; 
  
    return 0; 
} 
//Salida:
//Elements are : 10 11 12 13 14 15 16 17 18 19 20

//Codigo hecho con style=rrt
\end{pyglist}
\newpage
\subsection*{\texttt{bitset}}
Un \texttt{bitset} es un \texttt{array of bool}, pero cada valor booleano no se almacena por separado, en lugar de eso, \texttt{bitset} optimiza el espacio de tal manera que cada booleano toma 1 espacio de bits, por lo que el espacio que toma el \texttt{bitset} es menor de lo que tomaría un \texttt{array}. Podemos acceder a cada bit del \texttt{bitset} con la ayuda del operador de indexación de un \texttt{array,'[]'}, entonces \texttt{array[3]} muestra el bit en el índice 3 del \texttt{bitset}, recordando que se lee de derecha a izquierda y comienza en 0.
\begin{pyglist}[language=c++,caption={bitset},listingname={\textbf{Ejemplo}},style=perldoc]
#include <bits/stdc++.h> 
using namespace std; 
#define M 32 
int main() { 
    bitset<M> bset1;     // Constructor por default, establece todos los bits en 0 
    bitset<M> bset2(20); // bset2 es inicializado con los bits de 20 
    bitset<M> bset3(string("1100"));// bset3 es inicializado con los bits del string 
    cout << bset1 << endl; // 00000000000000000000000000000000 
    cout << bset2 << endl; // 00000000000000000000000000010100 
    cout << bset3 << endl; // 00000000000000000000000000001100 
    // declarando set8 con capacidad de 8 bits 
    bitset<8> set8; // 00000000 
    set8[1] = 1; // 00000010 
    set8[4] = set8[1]; // 00010010  
    int numberof1 = set8.count(); // count retorna el numero de 1's en el bitset 

    // size function retorna el numero total de bits en el bitset. 
    int numberof0 = set8.size() - numberof1; 
    cout << set8 << " has " << numberof1 << " ones and "
         << numberof0 << " zeros\n"; 

    // test function retorna 1 si el bit esta establecido, sino 0 
    cout << "bool representation of " << set8 << " : "; 
    
    for (int i = 0; i < set8.size(); i++) cout << set8.test(i) << " "; 
    // any function retorna true, si al menos 1 bit esta establecido.
    if (!set8.any()) cout << "set8 has no bit set.\n"; 
    // none function retorna true, si no hay ningun bit establecido. 
    if (!bset1.none()) cout << "bset1 has some bit set\n"; 
    
    cout << set8.set() << endl; //1 en todos los bits
    cout << set8.set(4, 0) << endl; //0 en el indice 4
    cout << set8.set(4) << endl; //1 en el indice 4
    cout << set8.reset(2) << endl; //0 en el indice 2 
    cout << set8.reset() << endl;  //0 en todo los bits
    cout << set8.flip(2) << endl; //flip en el indice 2
    cout << set8.flip() << endl; //flip en todos los bits
    // Converting decimal number to binary by using bitset 
    int num = 100; 
    cout << "\nDecimal number: " << num 
         << "  Binary equivalent: " << bitset<8>(num);
    return 0; 
} 
//Codigo hecho con style=perldoc
\end{pyglist}
\subsection*{\_\_builtin}
\subsubsection*{\texttt{\_\_builtin\_popcount}}
Esta función es usada para contar el número de 1's de un número:
\begin{pyglist}[language=c++,caption={popcount},listingname={\textbf{Ejemplo}},style=arduino]
#include <stdio.h> 
int main() { 
    int n = 5; 
    printf("Count of 1s in binary of %d is %d ",n, __builtin_popcount(n)); 
    //Count of 1s in binary of 5 is 2
}
//Codigo hecho con style=arduino
\end{pyglist}
\subsubsection*{\texttt{\_\_builtin\_parity}}
\begin{pyglist}[language=c++,caption={popcount},listingname={\textbf{Ejemplo}},style=tango]
#include <stdio.h> 
int main() { 
    int n = 7; 
    printf("Parity of %d is %d ",n, __builtin_parity(n)); 
    //Parity of 7 is 1
}
//Codigo hecho con style=tango 
\end{pyglist}
\subsubsection*{\texttt{\_\_builtin\_clz}}
Cuenta el número de 0's después del primer 1.
\begin{pyglist}[language=c++,caption={popcount},listingname={\textbf{Ejemplo}},style=emacs]
int main() { 
    int n = 16; 
    printf("Count of leading zeros before 1 in %d is %d",n, __builtin_clz(n));  
    //Count of leading zeros before 1 in 16 is 27
}
//Codigo hecho con style=emacs
\end{pyglist}
\subsubsection*{\texttt{\_\_builtin\_ctz}}
Cuenta el número de 0's hasta el primer 1.
\begin{pyglist}[language=c++,caption={popcount},listingname={\textbf{Ejemplo}},style=friendly]
int main() { 
    int n = 16; 
    printf("Count zeros from last to first occurrence of one is %d",builtin_ctz(n));
    //Count zeros from last to first occurrence of one is 4
} 
//Codigo hecho con style=friendly
\end{pyglist}
\subsection*{\texttt{stable\_sort}}
Funciona igual que \texttt{sort}, solo que si encontrase elementos iguales al comparar; los muestra manteniendo el orden.
\begin{pyglist}[language=c++,caption={bitset},listingname={\textbf{Ejemplo}},style=borland]
#include <iostream>
#include <algorithm>
#include <vector>
using namespace std;
bool compare_as_ints (double i,double j){
  return (int(i)<int(j));
}

int main () {
  double mydoubles[] = {3.14, 1.41, 2.72, 4.67, 1.73, 1.32, 1.62, 2.58};
  vector<double> myvector;
  myvector.assign(mydoubles,mydoubles+8);
  cout << "using default comparison:";
  stable_sort (myvector.begin(), myvector.end());
  for (vector<double>::iterator it=myvector.begin(); it!=myvector.end(); ++it)
    cout << ' ' << *it;
  cout << '\n';
  myvector.assign(mydoubles,mydoubles+8);
  cout << "using 'compare_as_ints' :";
  stable_sort (myvector.begin(), myvector.end(), compare_as_ints);
  for (vector<double>::iterator it=myvector.begin(); it!=myvector.end(); ++it)
    cout << ' ' << *it;
  cout << '\n';
  return 0;
}
//Salida
//using default comparison: 1.32 1.41 1.62 1.73 2.58 2.72 3.14 4.67
//using compare\_as\_ints: 1.41 1.73 1.32 1.62 2.72 2.58 3.14 4.67

//Codigo hecho con style=borland
\end{pyglist}
\end{document}
